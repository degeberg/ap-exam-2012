% vim: ft=tex

\section{Parsing Soil}

The Soil parser is implemented in the \verb|SoilParser| module (see
\autoref{source:parser}) using the ReadP parser combinator library.

The implementation is straightforward and is mostly done by a direct
translation from the grammar by creating a parser combinator for each
nonterminal. This makes verifying correctness easy as there is a clear
connection between the formal grammar and the implementation in Haskell. The
parser for the $Prim$ nonterminal stands out in this regard because it was
necessary eliminating the left-recursion in the grammar.

The chosen type for \verb|Error| is \verb|String|. However, ReadP has no
support for error messages---unlike e.g.\ Parsec has---so there is no good
choice of error message. Failure could just as well have been modelled using
the \verb|Maybe| monad, but this would violate the specified API.

\subsection{Testing}
\label{parser_testing}

The parser has been testing by unit testing using QuickCheck. Although
``proper'' properties have not been used, QuickCheck was chosen over HUnit
because it would make it easier adding more interesting properties later
without having to switch test framework.

The unit tests are simply checking the output of \verb|parseString| on a number
of programs with an expected AST. These were manually verified.
\verb|parseFile| is not tested automatically, but the definition is so simple
that it is easy to verify its correctness manually.
