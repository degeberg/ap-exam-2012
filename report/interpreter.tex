% vim: ft=tex

\section{Interpreting Soil}

The interpreter is implemented in the \verb|SoilInterp| module.

\subsection{Name and function environments}

The environments are simply \verb|Data.Map.Map|s. This makes lookups and
inserts easy in $O(\log n)$ time.

\subsection{Process queues}

The process queue is a simple association list with process IDs as the key.
Most of the utility functions are in a \verb|State| monad called
\verb|ProcessState|. The following things are in the state:

\begin{enumerate}
    \item Name environment
    \item Function environment
    \item Process queue
    \item Currently executing PID
    \item stdout
    \item stderr
\end{enumerate}

These things are useful when interpreting because the process queue utility
functions are in monadic style simply updating the state. These are all simple
one-line functions using the helper functions \verb|pqOp| and \verb|procOp|.
\verb|pqOp| applies a function to the process having a matching PID.
\verb|procOp| does this in monadic style with a state monad.

\subsection{Executing a process step}

When executing a process step, we first set the name environment and make sure
we remove the process' message. It is assumed that any PID given to
\verb|processStep| actually has a message. This should be taken care of by the
scheduler. This is slightly different from the suggested outline that states it
should simply do nothing.

Interpreting the process function's body is done by \verb|interpExpr|. This is
on simple pattern matching on the AST. This uses \verb|evalPrim| and
\verb|interpAct| to interpret the remaining data types in the AST.

If the process is one of the special \verb|#println| or \verb|#errorlog|
processes it updates the stdout and stderr state with whatever is in the inbox.

\subsection{Round-robin scheduling}

Round-robin scheduling just takes the first process with a non-empty mailbox
moving each to the back of the queue. When interpreting, we must first
initialize the state by populating the function environment and inserting two
dummy processes for \verb|#println| and \verb|#errorlog|.

\subsection{Nondeterministic process scheduling}
\label{nondet}

I was unable to complete the implementation of \verb|runProgAll|. An attempt is
in \verb|SoilInterp.hs|, but it doesn't work correctly\footnote{Although it
does type check, so according to Ken it must be correct.}. It always returns a
list with only one state.

\subsection{Testing}

Testing is done using QuickCheck with the same justification as in
\autoref{parser_testing}. This is in the \verb|SoilInterpTest.hs| file. The
test suite has a test for each example program and each scheduler. The three
tests for \verb|runProgAll| obviously fail as it's not implemented correctly as
explained in \autoref{nondet}.

The interpreter works correctly with round-robin on all the example programs.
It is then reasonable to assume that the interpreter works correctly.
