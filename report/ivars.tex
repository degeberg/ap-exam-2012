% vim: ft=tex

\section{IVars in Erlang}

The IVars have been implemented in the \verb|pm| module. An IVar is represented
by an Erlang process, operations on an IVar work by simple message passing.

They are implemented with two state functions
\verb|{vanilla,princess}_{set,unset}| giving a simple state machine. When an
IVar gets set it changes to work on its \verb|set| function. This design makes
it easy to distinguish behaviour on set and unset IVars.

Invalid predicates in princess IVars are handled using a \verb|try-catch| so
any exception will result in \verb|false|, and all non-\verb|true| values are
considered \verb|false| likewise.

\subsection{The \texttt{pmuse} module}
\label{module_pmuse}

The \verb|pmmap/2| and \verb|treeforall/2| functions are implemented in the
\verb|pmuse| module.

\verb|pmmap/2| initializes a vanilla IVar per element in the list and spawns a
new process putting the computed value in the IVar. This spawn gives the
concurrency as the function value can be computed in parallel. Then we just get
all the values in order. This ensures that order is preserved like in the
built-in \verb|lists:map/2|.

Assuming perfect concurrency, this would mean that mapping the function $f$
over a list $A$ takes $\max_{x \in A} T(f(x))$ time where $T(e)$ is the runtime
of executing some Erlang expression $e$. We can verify that the execution is
concurrent by running the program in \autoref{pmusecon}. Assuming we are able
to evaluate two things concurrently, this should take about 10 seconds in
total, whereas the normal, non-concurrent \verb|lists:map/2| would take about
20 seconds.

\begin{lstlisting}[language=Erlang,caption={Testing concurrency of \texttt{pmmap}},label=pmusecon]
pmuse:pmmap(fun(X) -> timer:sleep(10000), X end, [1, 2]).
\end{lstlisting}

The \verb|treeforall/2| function is implemented using princess IVars. First we
modify the predicate to also send a message back with the result of the
predicate check. Then we traverse the tree starting a princess IVar for each
node with this modified predicate, and we count the number of nodes. Knowing
the number of nodes we know the maximum number of results we receive. When
gathering up the result we know the result immediately if we get a \verb|false|
and just return. We can verify \verb|treeforall/2|'s short-circuit semantics
using \autoref{treeforallcon}. If short-circuiting works, this will return
\verb|false| immediately, and if it doesn't work, it would return after about
10 seconds.

\begin{lstlisting}[language=Erlang,caption={Testing short-circuit semantics of \texttt{treeforall}},label=treeforallcon]
pmuse:treeforall(
    {node, 1, {node, 2, leaf, leaf}, leaf},
    fun(X) ->
        case X of
            1 -> timer:sleep(10000), true;
            2 -> false
        end
    end
).
\end{lstlisting}

\subsection{Testing}

The modules have been tested using EUnit. The test suites may be run by calling
\verb|pm:test/0| and \verb|pmuse:test/0|. The following tests are executed:

\begin{enumerate}
    \item \texttt{pm}
    \begin{enumerate}
        \item Princess IVar
        \begin{enumerate}
            \item Unset variables are not compromised.
            \item Set variables are not compromised.
            \item Variables attempted to be set twice are not compromised.
            \item Values cannot be changed.
            \item Only a value satisfying the predicate is saved.
            \item Failing predicates work like \verb|false|.
        \end{enumerate}
        \item Vanilla IVar
        \begin{enumerate}
            \item Unset variables are not compromised.
            \item Variables are not compromised by being set.
            \item Variables are compromised by being set twice.
            \item Values cannot be changed.
        \end{enumerate}
    \end{enumerate}
    \item \texttt{pmuse}
    \begin{enumerate}
        \item \texttt{pmmap/2}
        \begin{enumerate}
            \item Mapping over the empty list is the empty list.
            \item Mapping with the identity function over a non-empty doesn't change the list.
            \item Order is preserved.
        \end{enumerate}
        \item \texttt{treeforall/2}
        \begin{enumerate}
            \item The empty tree satisfies any predicate.
            \item A singleton tree is satisfied by an always true predicate.
            \item A singleton tree is not satisfied by an always false predicate.
            \item Any node not satisfying a predicate will make the entire tree fail.
            \item A tree with multiple nodes all satisfying the predicate is satisfied.
        \end{enumerate}
    \end{enumerate}
\end{enumerate}

These tests along with the programs in \autoref{module_pmuse} verifying
concurrency and short-circuit semantics for \verb|pmmap/2| and
\verb|treeforall/2|, respectively, give a reasonable assurance that the
modules' functionality is correct.
